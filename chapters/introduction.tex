\chapter{Introduction}
\section{Report introduction}
In recent years, there has been a rapid proliferation of devices equipped with microphones, such as smartphones, smart speakers, and smartwatches.
This has led to a significant decline in user privacy, especially as many of these devices come with an 'always-listening' feature enabled by default.
As a result, the security vulnerabilities in microphones have attracted the attention of both attackers and defenders: while the motivations of the former are self-evident, the latter, in an interesting role reversal, can leverage these vulnerabilities to enhance user privacy.
In other words, it is possible to execute an attack that disables nearby microphones to preserve the privacy of users who may be unaware of potential eavesdropping.\\
\hspace*{2em}From the perspective of improving user privacy by exploiting microphones' vulnerabilities, this project investigates Denial of Service (DoS) attacks on microphones, with the goal of understanding the underlying technologies, identifying vulnerabilities, and replicating attacks on real hardware.
The project seeks to provide a comprehensive analysis of microphone manufacturing technologies, evaluate the state-of-the-art in attack techniques, and explore the limitations and implications of these attacks.
In particular, the project will focus on a specific type of DoS attack, which involves flooding the target microphone with ultrasound waves to prevent it from recording any conversation.
Additionally, a fully functioning prototype device capable of executing the attack has been developed.
As we navigate through both the sotware and hardware implementations of the project, an analysis of the results will be presented, assessing the strengths and weaknesses of the prototype device.
%Moreover, a prototype device has been developed, in order to demonstrate the effectiveness of the approach and assess its strengths and weaknesses.
\section{Report description}
The remainder of the document is organized as follows:
\begin{itemize}
    \item Chapter 2: This chapter offers a comprehensive overview of the project's theoretical background, detailing the key features of various microphone technologies and examining known attack methods.
    \item Chapter 3: This chapter provide a general overview of the project's implementation, offering the reader a simple description of the prototype device's components and capabilities.
    \item Chapter 4: This chapter presents a detailed explanation of the device, covering everything from its initial design to its physical construction and realization.
    \item Chapter 5: This chapter outlines the results obtained and proposes ideas for potential improvements to the device in future work.
    \item Chapter 6: This chapter recaps with a high-level summary what has been done in the project.
\end{itemize}

